\documentclass{sigchi}

% Use this command to override the default ACM copyright statement
% (e.g. for preprints).  Consult the conference website for the
% camera-ready copyright statement.


%% EXAMPLE BEGIN -- HOW TO OVERRIDE THE DEFAULT COPYRIGHT STRIP -- (July 22, 2013 - Paul Baumann)
% \toappear{Permission to make digital or hard copies of all or part of this work for personal or classroom use is      granted without fee provided that copies are not made or distributed for profit or commercial advantage and that copies bear this notice and the full citation on the first page. Copyrights for components of this work owned by others than ACM must be honored. Abstracting with credit is permitted. To copy otherwise, or republish, to post on servers or to redistribute to lists, requires prior specific permission and/or a fee. Request permissions from permissions@acm.org. \\
% {\emph{CHI'14}}, April 26--May 1, 2014, Toronto, Canada. \\
% Copyright \copyright~2014 ACM ISBN/14/04...\$15.00. \\
% DOI string from ACM form confirmation}
%% EXAMPLE END -- HOW TO OVERRIDE THE DEFAULT COPYRIGHT STRIP -- (July 22, 2013 - Paul Baumann)


% Arabic page numbers for submission.  Remove this line to eliminate
% page numbers for the camera ready copy

%\pagenumbering{arabic}

% Load basic packages
\usepackage{balance}  % to better equalize the last page
\usepackage{graphics} % for EPS, load graphicx instead
%\usepackage[T1]{fontenc}
\usepackage{txfonts}
\usepackage{times}    % comment if you want LaTeX's default font
\usepackage[pdftex]{hyperref}
% \usepackage{url}      % llt: nicely formatted URLs
\usepackage{color}
\usepackage{textcomp}
\usepackage{booktabs}
\usepackage{ccicons}
\usepackage{color,soul}
\usepackage{listings}

% llt: Define a global style for URLs, rather that the default one
\makeatletter
\def\url@leostyle{%
  \@ifundefined{selectfont}{\def\UrlFont{\sf}}{\def\UrlFont{\small\bf\ttfamily}}}
\makeatother
\urlstyle{leo}

% To make various LaTeX processors do the right thing with page size.
\def\pprw{8.5in}
\def\pprh{11in}
\special{papersize=\pprw,\pprh}
\setlength{\paperwidth}{\pprw}
\setlength{\paperheight}{\pprh}
\setlength{\pdfpagewidth}{\pprw}
\setlength{\pdfpageheight}{\pprh}

% Make sure hyperref comes last of your loaded packages, to give it a
% fighting chance of not being over-written, since its job is to
% redefine many LaTeX commands.
\definecolor{linkColor}{RGB}{6,125,233}
\hypersetup{%
  pdftitle={SIGCHI Conference Proceedings Format},
  pdfauthor={LaTeX},
  pdfkeywords={SIGCHI, proceedings, archival format},
  bookmarksnumbered,
  pdfstartview={FitH},
  colorlinks,
  citecolor=black,
  filecolor=black,
  linkcolor=black,
  urlcolor=linkColor,
  breaklinks=true,
}

% create a shortcut to typeset table headings
% \newcommand\tabhead[1]{\small\textbf{#1}}

% End of preamble. Here it comes the document.
\begin{document}

\title{EyeJS: A Framework for Generic Eye-Tracking Interactivity for Web Browsers and Node.JS}

\numberofauthors{2}
\author{%
  \alignauthor{Joshua R. Toenyes\\
    \affaddr{Dept. of Computer Science and Engineering}\\
    \affaddr{University of California San Diego}\\
    \affaddr{La Jolla, California, USA}\\
    \email{joshua.r.toenyes@jacobs.ucsd.edu}}\\
  \alignauthor{Nadir Weibel\\
    \affaddr{Dept. of Computer Science and Engineering}\\
    \affaddr{University of California San Diego}\\
    \affaddr{La Jolla, California, USA}\\
    \email{weibel@ucsd.edu}}\\
}

\maketitle

\begin{abstract}
  150 words...
\end{abstract}

\keywords{eye tracking; eye events; gaze interaction; blink interaction}

\category{H.5.2}{User Interfaces}{Input devices and strategies,
    Interaction styles (gaze and blinks), Screen design}
\category{H.5.4}{Hypertext/Hypermedia}{Architectures, Navigation}

\section{Introduction}
... We developed the framework using two eye trackers, Tobii's EyeX
tracker and The Eye Tribe's same-named eye tracker.

\section{Related Work}

\section{Background...}



\section{EyeJS DOM Events}
EyeJS provides a number of DOM events which may be listened-for in
the same manner as a native DOM event (such as \textit{mouseover}, \textit{mousemouse}, \textit{mouseout}, etc.). All EyeJS events
"bubble up" from the source DOM Element \hl{[definition and reference]},
so an event listener attached to some element will be called if the
corresponding event is triggered on the element itself or any of its
children (this is the default behavior of native events \hl{[check
if this isn't true for any events]}). Eye events provided by EyeJS
are described below.

\subsection{\textbf{\textit{gaze}} Event}
The \textit{gaze} event is fired whenever the user's gaze is over some
element. This event is most analogous to a \textit{mousemove} event.
Like a \textit{mousemove}, the user may actually be unaware of the
element over-which the event is triggered. For example, if the user's
eyes are currently executing a saccade while simultaneously being
sampled by the eye-tracker, the user will be unaware of the element they
are gazing at. \hl{[Lots of references about saccades and unawareness.]}
For this reason the \textit{fixation} event is provided (see below).

\subsection{\textbf{\textit{gazeleave}} Event}
The \textit{gazeleave} event is fired on an element whenever the 
user's gaze transitions from the target element (the one firing the
\textit{gazeleave}) to another element. The event is analogous to a \textit{mouseleave}.

\subsection{\textbf{\textit{fixation}} Event}
This event occurs when the user's gaze has remained fixated on a
single element for at least 110 milliseconds \hl{[citation about 
longest saccade duration]}. Since the sample rate of eye trackers 
can vary, we require consecutive samples of the gaze remaining over a 
particular element for at least 110 milliseconds to trigger this event. 
Using this timing we are reasonably sure the user is consciously aware 
of the element they are gazing at, and that the series of sample
eye-tracker sample points were not taken during a saccade. (Assuming 
the element is of reasonable size of course, as an element the size 
of the entire screen would constantly trigger a \textit{fixation} 
event regardless of the movement of their gaze.)

\subsection{\textbf{\textit{fixationend}} Event}
The \textit{fixationend} event is triggered on an element when the 
user's gaze transitions away from the subject element. This event is
similar to the \textit{gazeleave} event (and indeed a \textit{gazeleave}
event is triggered in tandem with a \textit{fixationend}), however
it is only triggered when a full fixation occurred prior to the user's
gaze departing the subject element.

\subsection{\textbf{\textit{blink}} Event}
Eyes opening and closing is a constant reality of human beings. We
constantly blink, but usually quiet quickly. The purpose of the 
\textit{blink} event is to detect a purposeful and conscious blink. 
In the author's [opinion?] this was best achieved by timing a blink
to be longer than an unconscious blink, but not long enough to be 
triggered by the eyes leaving the tracker's field of view. \hl{insert
sentence about blink timing here.} Using this event, complete 
hands-free use of the web-browser is possible.

\subsection{\textbf{\textit{eyesopen}} Event}
The \textit{eyesopen} event is triggered whenever both eyes transition
from being untracked (or unavailable) to the eye tracker, to both
being tracked (or available).

\subsection{\textbf{\textit{eyesclose}} Event}
Similar to the \textit{eyesopen} event, the \textit{eyesclose} event
is triggered when both eyes transition from being open to closed, as
measured by their tracking availability by the eye-tracker device.


\section{EyeJS Framework Components}
EyeJS is a framework which endeavors to bring eye-interactivity and
eye-tracking data to the web browser and Node.JS using a low-cost eye
tracker device. The framework is composed of three layers: first
an interface with an eye-tracking device, second a WebSocket server \hl{[WebSocket reference]} which provides raw gaze and eye availability
status data in a standardized format on a known local port, and finally
a JavaScript library for processing the eye-tracking data and
translating it into meaningful events in the web browser's Document
Object Model (DOM) \hl{[citation]}. We also developed a Google
Chrome Web Browser \hl{[citation?]} extension to enable
eye-interactivity on all web sites \hl{[HTTPs only?]} by injecting
the JavaScript library at page load.


\subsection{Eye Tracker Interface}
Eye tracking device manufacturers generally provide some type of SDK or
interface for communicating with the device. During the course of
development, we experimented with eye trackers from two manufacturers,
The Eye Tribe's self named eye tracker and the EyeX tracker from Tobii.
Each tracker has their own compatibilities and communication interfaces.
The eye tracker interface portion of EyeJS gathers tracker-provided
screen x and y gaze coordinates, and eye-availability (wether the
eye-tracker is currently tracking both, only the left or right, or
neither eyes).

\subsubsection{The Eye Tribe Tracker}
The Eye Tribe tracker is compatible with Windows 7 and 8 and Apple's Mac
OS X with Google Android OS support in-development. SDKs are provided
in C\#, C++ and Java, and it also has a well documented API for
communication via a TCP interface using JSON formatted messages. We
opted to develop a JavaScript SDK and Node.JS application to communicate
with this device using its TCP API. This allowed all three layers
(eye-tracker communication, WebSocket server, and on-page JS library)
to all be written in JavaScript. This device provides raw un-smoothed
screen gaze coordinates.

\subsubsection{EyeX Tracker}
Tobii's EyeX tracker is only compatible with Windows 7 and 8 and provides
SDKs for .NET, C/C++, and Unity. No other type of interface is offered
or documented making user-built SDKs in other languages impossible. To communicate with the EyeX device, we developed a lightweight C++
application. The EyeX tracker performs a small amount of smoothing on the
screen coordinates prior to providing them in software, although there
is an additional SDK available (the Tobii Gaze SDK) which provides access
to the raw, un-smoothed data. We did not use this SDK for EyeJS.


\subsection{WebSocket and EyeJS Subprotocol}
After the screen gaze coordinates are collected from the eye-tracker,
they must be made available for a web-browser running on the same host.
This is accomplished using a WebSocket server which provides lightweight
low-latency 2-way communication between the web-browser and the server.
The WebSocket server provides an interface running on a known port, 5225 \hl{[check the port]}, which the JavaScript library running on a web-page
connects to.

\subsubsection{EyeJS WebSocket Subprotocol}
EyeJS communication currently occurs in only one direction, from
the WebSocket server to client web-browsers. We did not implement
any eye-tracker commands. The format of these messages are well
defined and standardized
for both the Tobii EyeX tracker and The Eye Tribe trackers. Conceivably,
the EyeJS Framework could be extended to include other eye-trackers,
which would only need to provided a WebSocket interface implementing
this subprotocol. Each message is formatted as follows:

\begin{lstlisting}
{
  // event type
  type: 'gaze',

  // millisecond timestamp
  timestamp: < number >,

  // mean gaze screen-coordinates of
  // both left and right eyes
  avg: {
    x: < number >,
    y: < number >
  },

  // left-eye gaze screen-coordinates
  // of only left eye
  left: {
    x: < number >,
    y: < number >
  },

  // right-eye gaze screen-coordinates
  // of only left eye
  right: {
    x: < number >,
    y: < number >
  }

  // true if the indicated eye is being
  // tracked by the eye-tracker
  available: {
    left:   < boolean >,
    right:  < boolean >,
    both:   < boolean >
  }
}
\end{lstlisting}


\subsection{EyeJS JavaScript Library}
The JavaScript library processes and converts the eye tracker data
into meaningful events within the web-browser.

\subsubsection{Data Smoothing}

\subsubsection{}



\section{Evaluation}

\subsection{Eye Tracker Evaluation}
... The Eye Tribe API proved especially valuable for us, as we were
able to build a Node.JS application to communicate to the eye tracker
...

\subsection{Description of Tasks}

\subsection{Results}

\section{Discussion}

\subsection{Limitations}

\subsection{Future Work}

\subsubsection{Custom Calibration Interface}
The Eye Tribe tracker provides API access to calibration routines, so
it is conceivable that in the future EyeJS could be extended to include
a standardized user interface for calibrating the eye tracker.

\subsubsection{Continuous Calibration}
As eye trackers advance and presumably provide a more advanced interface
for developers and applications, it could be possible to perform
continuous calibration improvement of the eye-tracker by inferring
interactions... activate some element therefore the user was probably
looking at that element, therefore push a calibration offset point...

\section{Conclusion}

\section{Acknowledgments}

\balance{}

% REFERENCES FORMAT
\bibliographystyle{sigchi}
\bibliography{eyejs}

\end{document}

%%% Local Variables:
%%% mode: latex
%%% TeX-master: t
%%% End:
