\section{Background}
We want access to where the user is looking for research tasks, inferring attention \hl{reference about eye as indicator of attention}, and perhaps measuring user engagement.

Some users could use it as an form of input... specifically paraplegics and in the extreme case LIS patients.

...

While eye tracking may indeed be a useful input device, it is also equally a an important form of implicit user input. It has been shown that user attention can be measured using eye tracking \hl{ref}.

...

Low-cost eye tracking is still not perfect and there remains significant room for improvement. However, if web designers and developers are to remain current, we need to start thinking about how these devices will be used to interact with our websites and web applications in the near future. Similarly, browser vendors should start seriously considering native support for these input devices.

Naturally, our eyes are used as a sensory devices. Things in the world don't normally move or react as a result of us gazing at them. We use our eyes to gather information about our surroundings and hands or other more mechanical means as tools to manipulate our surroundings. So on a fundamental level, we can see why using our eyes to control something is inherently unnatural. \hl{Reference to support this.}

While eye control may seem unnatural and may not be the preferred primary form of input, there are many opportunities to use it .... Nevertheless, there are some users who lack the capability to

Suppose we were given access to gaze information, in addition to mouse events and keyboard events? Suppose we stopped trying to replicate the mouse experience with gaze information, and accepted that gaze and mouse inputs are inherently different? What could we accomplish given this additional information? How could we build websites and web applications to take advantage of this information? Could we build interfaces that are entirely intent driven, instead of mechanically driven?



The goal of EyeJS is threefold: first to enable generic web navigation on existing websites using low-cost consumer grade eye trackers, second to contribute to the discussion of possible future standardization of web browser eye events, and third to give web developers access to eye events (for users with eye trackers) to enable the use of eye trackers in web apps and Node.JS based desktop application.



----------

Figure \ref{fig:keypress-accuracy} shows an increase in accuracy at the smallest button size. This is due to the fact that not all users were able to complete the selection task (select a button) before the task timer ran out. \hl{Insert plot of the number of users who completed each input mode at each button size.}
